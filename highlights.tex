\documentclass[11pt]{article}
\usepackage[utf8]{inputenc}
\usepackage[T1]{fontenc}
\usepackage{amsmath}
\usepackage[margin=1in]{geometry}
\usepackage{times}
\usepackage{enumitem}
\usepackage{xcolor}

\title{\textbf{Research Highlights}}
\author{}
\date{}

\begin{document}

\maketitle

\vspace{-1cm}

\textbf{First systematic comparative analysis of lattice geometries in locally resonant sonic crystal plates: PWE/EPWE computational framework with finite element validation}

\vspace{0.5cm}

\begin{itemize}[leftmargin=0.5cm, itemsep=0.3cm]

\item \textcolor{red}{\textbf{First systematic comparative analysis} of five lattice geometries (square, rectangular, triangular, honeycomb, kagomé) for locally resonant metamaterial plates using validated PWE/EPWE framework.}

\item \textcolor{red}{\textbf{Systematic bandwidth evolution mapping} across 15 resonator frequencies (10-150 Hz) reveals geometry-dependent optimal operational ranges and establishes frequency-dependent performance maps.}

\item \textcolor{red}{\textbf{Dual bandgap characterization in multi-resonator systems:} Honeycomb/kagomé achieve broadband multi-frequency attenuation through in-phase and anti-phase resonator coupling modes.}

\item \textcolor{red}{\textbf{Quantitative performance hierarchy:} Triangular lattices achieve 35\% superior relative bandwidth (42.51\% vs 31.40\%) using only 25\% of kagomé material; computational efficiency: 1800-5700× speedup over FEM.}

\item \textcolor{red}{\textbf{Engineering design framework} with frequency-dependent lattice selection guidelines for aerospace, automotive, and civil vibration control applications.}

\end{itemize}

\vspace{0.5cm}

\textbf{Keywords:} Locally resonant metamaterial, Flexural waves, Band gaps, Lattice configurations, Semi-analytical method, Frequency-dependent optimization, Low-frequency vibration control

\end{document}