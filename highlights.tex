\documentclass[11pt]{article}
\usepackage[utf8]{inputenc}
\usepackage[T1]{fontenc}
\usepackage{amsmath}
\usepackage[margin=1in]{geometry}
\usepackage{times}
\usepackage{enumitem}
\usepackage{xcolor}

\title{\textbf{Research Highlights}}
\author{}
\date{}

\begin{document}

\maketitle

\vspace{-1cm}

\textbf{First systematic comparative analysis of lattice geometries in locally resonant sonic crystal plates: PWE/EPWE computational framework with finite element validation}

\vspace{0.5cm}

\begin{itemize}[leftmargin=0.5cm, itemsep=0.3cm]

\item \textcolor{red}{\textbf{First systematic comparative analysis} of five lattice geometries (square, rectangular, triangular, honeycomb, kagomé) for locally resonant metamaterial plates using validated PWE/EPWE framework.}

\item \textcolor{red}{\textbf{Computational efficiency breakthrough:} Semi-analytical methods achieve 1800-5700× speedup over FEM while maintaining <1\% error, enabling large-scale optimization.}

\item \textcolor{red}{\textbf{Quantitative performance hierarchy established:} Triangular lattices achieve 35\% superior relative bandwidth (42.51\% vs 31.40\%); kagomé provides 15 dB enhanced low-frequency attenuation; finite plates show 40-50\% bandwidth expansion.}

\item \textcolor{red}{\textbf{Material efficiency optimization:} Triangular lattices achieve superior performance using only 25\% of kagomé's material, providing quantitative geometry-performance trade-offs.}

\item \textcolor{red}{\textbf{Engineering design framework:} First comprehensive decision matrix with frequency-dependent lattice selection guidelines for aerospace, automotive, and civil applications.}

\end{itemize}

\vspace{0.5cm}

\textbf{Keywords:} Locally resonant metamaterial, Flexural waves, Band gaps, Lattice configurations, Semi-analytical method, Frequency-dependent optimization, Low-frequency vibration control

\end{document}