\documentclass{article}
\usepackage{graphicx}
\usepackage{subcaption}
\usepackage{xcolor}
\usepackage{tikz}

\begin{document}

\begin{figure*}[t]
\centering

% Linha superior - Subfiguras a), b), c)
\begin{subfigure}[b]{0.31\textwidth}
    \centering
    \includegraphics[width=\textwidth]{figura_a_disp_hex_SEM_LEGENDA.png}
    \caption{Dispersion diagram}
    \label{fig:disp_epwe_hex}
\end{subfigure}
\hfill
\begin{subfigure}[b]{0.31\textwidth}
    \centering
    \includegraphics[width=\textwidth]{figura_b_impedance_hex_SEM_LEGENDA.png}
    \caption{Impedance}
    \label{fig:impedance_hex}
\end{subfigure}
\hfill
\begin{subfigure}[b]{0.31\textwidth}
    \centering
    \includegraphics[width=\textwidth]{figura_c_frf_hex_SEM_LEGENDA.png}
    \caption{Receptance FRF}
    \label{fig:frf_hex}
\end{subfigure}

\vspace{0.2cm}

% Legenda unificada para a), b), c) - COM DUAS BANDAS
\centering
\small
\begin{tabular}{@{}c@{\hspace{0.3em}}l@{\hspace{1.0em}}c@{\hspace{0.3em}}l@{\hspace{1.0em}}c@{\hspace{0.3em}}l@{}}
% Linha 1 - Bandas proibidas e primeiros modos
\tikz{\filldraw[magenta!90!red] (0,0) rectangle (0.6,0.3);} & FBGW 1 &
\tikz{\filldraw[orange!90!red] (0,0) rectangle (0.6,0.3);} & FBGW 2 &
\tikz{\draw[line width=3.5pt, orange!90!yellow] (0,0.15) -- (0.6,0.15);} & K-L: Mode 1 - PWE \\[0.3em]

% Linha 2 - Modos 2, 3, 4
\tikz{\draw[line width=3.5pt, cyan!80!white] (0,0.15) -- (0.6,0.15);} & K-L: Mode 2 - PWE &
\tikz{\draw[line width=3.5pt, red!40!orange!60] (0,0.15) -- (0.6,0.15);} & K-L: Mode 3 - PWE &
\tikz{\draw[line width=3.5pt, blue!30!red!20] (0,0.15) -- (0.6,0.15);} & K-L: Mode 4 - PWE \\[0.3em]

% Linha 3 - EPWE, FRF e Lrsc
\tikz{\draw[line width=3pt, blue!70!black, dashed] (0,0.15) -- (0.6,0.15);} & K-L: Mode 1 - EPWE ($\Re$) &
\tikz{\draw[line width=3.5pt, orange!90!yellow] (0,0.15) -- (0.6,0.15);} & K-L: Mode 4 - EPWE ($\Im$) &
\tikz{\draw[line width=3.5pt, green!60!black] (0,0.15) -- (0.6,0.15);} & K-L: FRF - FEM \\[0.3em]

% Linha 4 - Lrsc
\tikz{\draw[line width=2.5pt, black, dash pattern=on 4pt off 2pt on 1pt off 2pt] (0,0.15) -- (0.6,0.15);} & Lrsc: 30 [Hz] & & & & \\
\end{tabular}

\vspace{0.4cm}

% Linha inferior - Subfiguras d), e), f)
\begin{subfigure}[b]{0.31\textwidth}
    \centering
    \includegraphics[width=\textwidth]{figura_d_mode_ah.png}
    \caption{$A_h$ at 18.40 Hz}
    \label{fig:mode_ah}
\end{subfigure}
\hfill
\begin{subfigure}[b]{0.31\textwidth}
    \centering
    \includegraphics[width=\textwidth]{figura_e_mode_bh.png}
    \caption{$B_h$ at 29.21 Hz}
    \label{fig:mode_bh}
\end{subfigure}
\hfill
\begin{subfigure}[b]{0.31\textwidth}
    \centering
    \includegraphics[width=\textwidth]{figura_f_mode_ch.png}
    \caption{$C_h$ at 38.37 Hz}
    \label{fig:mode_ch}
\end{subfigure}

\caption{Band structure, frequency response, and wave propagation modes for a hexagonal lattice phononic crystal with local resonators ($f_r = 30$ Hz). 
(\textit{a}) Dispersion diagram along $\Gamma$--X computed with PWE and enriched PWE (EPWE) methods showing FBGW 1 and FBGW 2. 
(\textit{b}) Imaginary part of the impedance from EPWE (Mode 4) showing resonance behavior. 
(\textit{c}) Receptance (point FRF) computed with FEM showing attenuation within the band gaps. Points $A_h$, $B_h$, and $C_h$ indicate selected frequencies for modal analysis. 
(\textit{d--f}) Displacement field patterns at frequencies corresponding to points $A_h$, $B_h$, and $C_h$, computed by FEM.}
\label{fig:epwe_frf_modes_hex}
\end{figure*}

\end{document}